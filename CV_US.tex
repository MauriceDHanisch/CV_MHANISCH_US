%% start of file `template.tex'.
%% Copyright 2006-2013 Xavier Danaux (xdanaux@gmail.com).
%
% This work may be distributed and/or modified under the
% conditions of the LaTeX Project Public License version 1.3c,
% available at http://www.latex-project.org/lppl/.


\documentclass[11pt,a4paper,roman]{moderncv}        % possible options include font size ('10pt', '11pt' and '12pt'), paper size ('a4paper', 'letterpaper', 'a5paper', 'legalpaper', 'executivepaper' and 'landscape') and font family ('sans' and 'roman')

% modern themes
\moderncvstyle{banking}                            % style options are 'casual' (default), 'classic', 'oldstyle' and 'banking'
\moderncvcolor{black}                                % color options 'blue' (default), 'orange', 'green', 'red', 'purple', 'grey' and 'black'
%\renewcommand{\familydefault}{\sfdefault}         % to set the default font; use '\sfdefault' for the default sans serif font, '\rmdefault' for the default roman one, or any tex font name
\nopagenumbers{}                                  % uncomment to suppress automatic page numbering for CVs longer than one page

% character encoding
\usepackage[utf8]{inputenc}
\usepackage{fontawesome5}
\usepackage{tabularx}
\usepackage{comment}
\usepackage{ragged2e}
\usepackage{graphicx}
\usepackage{tikz}
\usepackage{tikzpagenodes}
\usepackage{calc}
\usepackage[super]{nth}
\newcommand*{\ClipSep}{0.4cm}%
\usetikzlibrary{calc} 
\usepackage[absolute,overlay]{textpos}
\usepackage[
top    = 3.3cm,
bottom = 2cm,
left   = 1.9cm,
right  = 1.9cm]{geometry}

\usepackage{multicol}
\usepackage{import}
\usepackage{xcolor}

% personal data
\name{Maurice D.}{Hanisch}
\address{Gubelstrasse 44}{Zurich}{Switzerland}% optional, remove / comment the line if not 
  
\newcommand*{\customcventry}[5][.5em]{
  \begin{tabular}{@{}l} 
    {\bfseries #2}
  \end{tabular}
  \hfill% move it to the right
  \begin{tabular}{l@{}}
     { #3}
  \end{tabular} \\
  \begin{tabular}{@{}l} 
    {\itshape\small #4}
  \end{tabular}
  \ifx&#5&%
  \else{\\%
    \begin{minipage}{0.82\maincolumnwidth}%
      \small#5%
    \end{minipage}}\fi%
  \par\addvspace{#1}}

\newcommand*{\customcvproject}[4][.25em]{
%   \vfill\noindent
  \begin{tabular}{@{}l} 
    {\bfseries #2}
  \end{tabular}
  \hfill% move it to the right
  \begin{tabular}{l@{}}
     { #3}
  \end{tabular}
  \ifx&#4&%
  \else{\\%
    \begin{minipage}{0.82\maincolumnwidth}%
      \small#4%
    \end{minipage}}\fi%
  \par\addvspace{#1}}
  
\newcommand*{\custompublication}[2][0.6em]{
    \begin{minipage}{\maincolumnwidth}%
       #2%
    \end{minipage}
  \par\addvspace{#1}}

\newcommand{\blackitem}{\item[\large\color{black}\textbullet]}

\setlength{\tabcolsep}{12pt}

%----------------------------------------------------------------------------------
%            content
%----------------------------------------------------------------------------------
\begin{document}
\begin{textblock*}{14cm}(3.7cm,1.5cm) % {block width} (coords) 
    \makecvtitle
\end{textblock*}
\begin{textblock*}{15cm}(2.8cm,2.5cm)
\begin{center}
\begin{tabular}{ c c }
 \faEnvelope\enspace mhanisc@student.ethz.ch ~~ & \faMobile\enspace \quad +41 763 61 91 83 \\
\end{tabular}
\end{center}
\end{textblock*}
\begin{textblock*}{15cm}(2.8cm,2.9cm)
\begin{center}
\begin{tabular}{ c c }
 \faLinkedin\enspace \href{https://www.linkedin.com/in/mauricehanisch/}{\underline{www.linkedin.com/in/mauricehanisch/}} & \faGithub\enspace\href{https://github.com/MauriceDHanisch/}{\underline{www.github.com/MauriceDHanisch/}}\\
\end{tabular}
\end{center}
\end{textblock*}

\vspace*{0mm}

\section{Education}
\customcventry{M.Sc. Physics}{09/2022 -- Present}{ETH Zurich, Switzerland}
{\begin{itemize}
    \item Current GPA: 5.61/6.00 (CH)
    \item Focus: Q. Error Correction, Q. Information Processing, Machine Learning
    \item Selected past \& current coursework: Advanced Q. Algorithms, Probabilistic Artificial Intelligence, Q. Error Correction, Q. Information Processing 1 \& 2, Trapped Ions.
\end{itemize}
}
\vspace{-0.1cm}
{\textit{| Partly financed by the \underline{Rayer scholarship} | Dr. Mas-Fraissinet, France.}}
\vspace{0.3cm}

\customcventry{B.Sc. Physics}{10/2018 -- 09/2022}{LMU Munich, Germany}
{\begin{itemize}
  \item GPA: 1.28/1.00 (DE).
  \item Theoretical physics electives.
\end{itemize}
}
\vspace{-0.1cm}
{\textit{| Mainly financed by the \underline{Rayer scholarship} | Dr. Mas-Fraissinet, France.}}
\vspace{0.3cm}

\customcventry{Medical Degree}{10/2017 -- 02/2018}{LMU Munich, Germany}
{\begin{itemize}
  \item GPA: N/A. Not completed.
\end{itemize}
}

\customcventry{Baccalauréat}{09/2014 -- 07/2017}{Lycée Sud des Landes, St-Vincent-de-Tyrosse, France}
{\begin{itemize}
    \item GPA 16.00/20.00 (FR). Graduated with the highest possible honors at the age of 16.
\end{itemize}
}



%%%%%%%%%%%%%%%%%%%%%%%%%%%%%%%%%%%%%%
%%%%%%% RESEARCH EXPERIENCE %%%%%%%%%%
%%%%%%%%%%%%%%%%%%%%%%%%%%%%%%%%%%%%%%


\section{Research Experience}
\customcventry{Master's Thesis}{09/2023 -- Present}{IBM Quantum, Zurich, Switzerland}
{\begin{itemize}
  \item[] \hspace*{-0.38cm}{\scriptsize \href{https://github.com/MauriceDHanisch/soft_info}{\underline{\faGithub~Project Repo (still private)}}}
  \vspace*{0.2cm}
  \item \underline{``Quantum error correction with analog measurement information decoders.''}
  \item Writing soft information decoders in C\texttt{++} \& Python. Analyse performance on real hardware.
  \item Future work:  Analyzing performance-information trade-off of soft-information decoders. Working on decoders for ArcCircuits. 
  \item Skills: C\texttt{++}, Python, IBM Quantum hardware.
  \item Supervisors: Dr. James Wootton, Dr. Joseph Renes, Prof. Renato Renner.
\end{itemize}}

\vspace*{0.2cm}
\customcventry{Semester Project}{06/2023 -- 08/2022}{ETH Zurich, Switzerland}
{\begin{itemize}
    \item[] \hspace*{-0.38cm}{\scriptsize \href{https://ethz.ch/content/dam/ethz/special-interest/phys/quantum-electronics/tiqi-dam/documents/semester_theses/smesterthesis_Maurice_Hanisch}{\underline{\faLink~Report Link}}}
    \vspace*{0.2cm}
    \item \underline{``Towards a tunable beamsplitter interaction between two GKP-encoded qubits.''}
    \item Experimental project on the motional interaction of GKP-encoded qubits in trapped ions.
    \item Designed and simulated trapping potentials and experimental testing of the found potentials.
    \item Skills: Python, ion trapping hardware. 
    \item Supervisor: Dr. Stephan Welte, Moritz Fontbot\'e-Schmidt, Prof. Jonathan Home.
\end{itemize}}

\newpage

\customcventry{Bachelor's Thesis}{04/2022 -- 09/2022}{Max-Planck-Institute for Quantum Optics, Munich, Germany}
{\begin{itemize}
    \item[] \hspace*{-0.38cm}{\scriptsize \href{https://pure.mpg.de/pubman/faces/ViewItemFullPage.jsp?itemId=item_3458224}{\underline{\faLink~Report Link}}}
    \vspace*{0.2cm}
    \item \underline{``Space-efficient quantum computation of fermionic and bosonic Gaussian systems.''}
    \item Introduction to various theoretical research areas (entanglement simulation with Python, entropic uncertainty principles, tensor networks) by Dr. Adrian Rubio.
    \item Investigated the time and space complexity of fermionic and bosonic Gaussian circuits.
    \item Skills: Python, complexity theory, bosonic quantum computation.
    \item Supervisors: Dr. Adrian Rubio, Prof. Ignacio Cirac.
\end{itemize}}




%%%%%%%%%%%%%%%%%%%%%%%%%%%%%%%%%%%%%%
%%%%%%%%%%%%% TEACHING %%%%%%%%%%%%%%%
%%%%%%%%%%%%%%%%%%%%%%%%%%%%%%%%%%%%%%

\section{Teaching}
\customcventry{Teaching Assistant in Linear Algebra I}{09/2023 -- Present}{ETH Zurich, Switzerland}
{\begin{itemize}
  \item Teached Linear Algebra for \nth{1} year B.Sc. CS students in a weekly in-person class.
  \item Lecturer: Prof. Bernd Gärtner, Prof. Afonso Bandeira.
\end{itemize}}

\customcventry{Teaching Assistant in Physics II}{02/2023 -- 07/2023}{ETH Zurich, Switzerland}
{\begin{itemize}
  \item Teached Continuum Mechanics and Thermodynamics for \nth{1} year B.Sc. EE students in a weekly in-person class.
  \item Lecturer: Prof. Atac Imamoglu.
\end{itemize}}

\customcventry{Laboratory Supervisor}{10/2021 -- 02/2022}{LMU Munich, Germany}
{\begin{itemize}
  \item Supervised the physics electrodynamic laboratory course for five groups of 20 medicine students.
\end{itemize}}




%%%%%%%%%%%%%%%%%%%%%%%%%%%%%%%%%%%%%%
%%%%%%%%% WORK EXPERIENCE %%%%%%%%%%%%
%%%%%%%%%%%%%%%%%%%%%%%%%%%%%%%%%%%%%%

\section{Work Experience}
\customcventry{IP Law Intern}{04/2022}{Bardehle Pagenberg, Munich, Germany}
{\begin{itemize}
  \item Drafted argumentations for ongoing patent lawsuits. Familiarized with litigation examples. Supervisor: Sebastian H.-E. Müller.
\end{itemize}}

\customcventry{Assistance Coordinator}{02/2018 -- 10/2018}{IMA, Munich, Germany}
{\begin{itemize}
  \item Accident and breakdown assistance coordinator in regions including DE, CH, FR, and UK.
\end{itemize}}





%%%%%%%%%%%%%%%%%%%%%%%%%%%%%%%%%%%%%%
%%%%%%%%%% COMPETITIONS %%%%%%%%%%%%%%
%%%%%%%%%%%%%%%%%%%%%%%%%%%%%%%%%%%%%%

\section{Competitions}
\customcventry{Quandela Hackathon: LOQCathon 2.0}{11/2023}{Sorbonne Université, Paris, France}
{\begin{itemize}
    \item[] \hspace*{-0.38cm}{\scriptsize \href{https://github.com/MauriceDHanisch/LOQCathon_qReservoir}{\underline{\faGithub~Project Repo}}}
    \vspace*{0.05cm}
    \item Won overall \underline{\nth{1} Prize} of the competetion with my team (3000€).
    \item Topic: Quantum reservoir computing using only linear optical elements. 
\end{itemize}}


\customcventry{Qiskit Hackathon}{06/2023}{World of Photonics, Munich, Germany}
{\begin{itemize}
    \item Topic: Graph theory-based approach to encode highly entangled states on IBM hardware.
\end{itemize}}


\customcventry{ETH Quantum Hackathon}{05/2023}{ETH Zurich, Switzerland}
{\begin{itemize}
    \item Won \underline{\nth{2} Prize} of the IQM challenge with my team.
    \item Topic: Using symmetry-informed quantum machine learning.
\end{itemize}}




%%%%%%%%%%%%%%%%%%%%%%%%%%%%%%%%%%%%%%
%%%%%%%%%% EXTRACURRICULAR %%%%%%%%%%%
%%%%%%%%%%%%%%%%%%%%%%%%%%%%%%%%%%%%%%

\section{Extracurricular Activities}

\customcventry{Student Mentor}{09/2023 -- 02/2024}{ETH Physics Department Student Mentoring}
{\begin{itemize}
    \item Mentor for a group of seven first-year BSc students. Weekly meetings to discuss the student's progress and problems.
\end{itemize}}

\customcventry{Residence House Speaker}{10/2021 -- 09/2022}{Studentenwerk Munich, Germany}
{\begin{itemize}
  \item  One out of 2 speakers of a 300-student residence house. Coordinated administrative tasks. Organized student events. Mediated disputes and conflicts. Responsible for sensible issues.
\end{itemize}}

\customcventry{Student Mentor}{10/2021}{LMU Physics Department}
{\begin{itemize}
  \item Mentor 40 new LMU physics students to assist them in commencing their studies in the first months of their studies.
\end{itemize}
}





%%%%%%%%%%%%%%%%%%%%%%%%%%%%%%%%%%%%%%
%%%%%%% SKILLS & INTERESTS %%%%%%%%%%%
%%%%%%%%%%%%%%%%%%%%%%%%%%%%%%%%%%%%%%

\section{Technical Skills and Interests}
% TO DO: Python libraries aufzählen
\textbf{Technical:}
\begin{itemize}
\item \underline{Python:} qiskit, pymatching, stim, scikit-learn. 
\item \underline{C\texttt{++}:} pybind11, eigen.
\item \underline{Docker}
\item \underline{LaTeX}
\end{itemize}
\vspace*{0.1cm}
\textbf{Languages:} German (native), French (native), English (C1: IELTS 8/9)
\\
\textbf{Interests:} Calisthenics, Weightlifting, Volleyball, Chess.




\end{document}





